\documentclass[11pt]{article}
\usepackage[left=3cm,right=3cm]{geometry} 			% geometry of document
\usepackage{amsmath,amssymb,amsthm,amsfonts}		% maths
\usepackage{booktabs,tabularx}						% tables
\usepackage{listings,xcolor,textcomp}				% needed for fortran code
\lstset{language=[90]Fortran,
  basicstyle=\ttfamily,
  keywordstyle=\color{red},
  commentstyle=\color{blue},
  morecomment=[l]{!\ }% Comment only with space after !
}
\usepackage[parfill]{parskip} 						% new line instead of indentation

\newcommand{\pd}[2]{\frac{\partial #1}{\partial #2}}


\begin{document}
\section*{\texttt{WABBIT} style guide}
T.~Engels, P.~Krah, S.~Mutzel, J.~Rei\ss, M.~Sroka 
\\[1ex] 
\today
\\[2ex]
\hrule$~$

In the following we summarize some guidelines for writing source code in the \texttt{WABBIT}-software package. 
\subsection*{Formal Guidelines}
\begin{itemize}
    \item indent with 4 spaces
	\item limit the line length to 80 characters
	\item all strings are limited to 80 characters
	\item all programs contain \texttt{implicit none} to force declaration
\end{itemize}

\subsection*{Name conventions}
\begin{itemize}
	\item use \texttt{x,y,z} for coordinates of \texttt{type real}
	\item use \texttt{ix,iy,iz} for spatial index of \texttt{type integer}
	\item use \texttt{q\_x,q\_y,q\_z} for partial derivatives, i.e $q_x=\pd{q}{x}$
\end{itemize}

\begin{table}[htp!]
	\caption{summary of naming conventions}
	\label{tab:tablename}
	\centering

	\begin{tabular}{l>{\tt}l}
	\toprule
		\textbf{Names} & \textbf{Scheme} \\
	\midrule
	Keywords				&print, select case \\
	Variables				&time,  dissipation \\
	Parameters				&PI, E, GAMMA\_EULER\_MASCHERONI \\
	Procedures/Subroutines  &Timestepper, Distance\_to\_Obj \\
	Modules  				&module\_Time, module\_Mesh \\
	Typ 					&typ\_params, type\_circle\\
	\bottomrule
	\end{tabular}
\end{table}

\subsection*{Error messaging}

Use the following syntax for error messages:
\begin{lstlisting}
call abort(<UNIQUE_INTEGER>,"ERROR [<file>.f90]: my error message!")
\end{lstlisting}
\noindent
where \texttt{<UNIQUE\_INTEGER>} is an Integer you choose (e.g 65432) and \texttt{<file>.f90} the file containing the source code.

\subsection*{Documentation using \texttt{doxygen}}

	\begin{itemize}
		\item for every new subdirectory use 
		\begin{lstlisting}
!> \dir
!> \brief
!! Brief description of the subdirectory goes here
\end{lstlisting}
		in a main file to give a short description of the entire directory

	\item for every new file use 
		\begin{lstlisting}
!> \file
!> \brief
!! brief description of the module goes here
!> \details
!! detailed description goes here. Use additional elements
!!    * Markdown style (items or headers etc.)
!!    * use latex code with the help of \f$ E=mc^2\f$
!!    * include images: 
!!		\image html maskfunction.bmp "plot of chi(delta)"
!!    * Cite code with:
!!		<a href="your_URL/DOI">Author (2015)</a>
!!	
!> \version 0.5
!> \date 23.1.2018 creation of module (commit b1234)
!> \date 23.2.2018 subroutine Calculate_Timestep added (commit c14656)
!> \author M.Mustermann
!> \author S.Musterfrau
\end{lstlisting}
	\item for example on the beginning of the main module inside a subdirectory:
	
\begin{lstlisting}
!-----------------------------------------------------------------
!> \dir
!> \brief
!! Brief description of the subdirectory goes here
!-----------------------------------------------------------------
!> \file
!> \brief
!! brief description of the module goes here
!> \details
!! detailed description goes here. Use additional elements
!!    * Markdown style (items or headers etc.)
!!    * use latex code with the help of \f$ E=mc^2\f$
!!    * include images: 
!!		\image html maskfunction.bmp "plot of chi(delta)"
!!    * Cite code with:
!!		<a href="your_URL/DOI">Author (2015)</a>
!!	
!> \version 0.5
!> \date 23.1.2018 creation of module (commit b1234)
!> \date 23.2.2018 subroutine Calculate_Timestep added (commit c14656)
!> \author M.Mustermann
!> \author S.Musterfrau
!-----------------------------------------------------------------
\end{lstlisting}

	\item Documentation of functions 

\begin{lstlisting}
!-----------------------------------------------------------------
!> \file
!> \brief Right hand side for 2D navier stokes equation
!>        ---------------------------------------------
!> The right hand side of navier stokes in the skew symmetric form 
!> is implemented as follows:
!>\f{eqnarray*}{
!!     \partial_t \sqrt{\rho} &=& ...
!!\f}
!!
!> \version 0.5
!> \date 08/12/16 - create \n
!> \date 13/2/18 - include mask and sponge terms (commit 1cf9d2d53ea76e)
!!
!> \author Pkrah
!-----------------------------------------------------------------


!>\brief main function of RHS_2D_navier_stokes
subroutine Rhs_NS_2D( parameter)
    implicit none

    !> parameter description
    integer(kind=ik), intent(in)     :: parameter 
    ...

end subroutine Rhs_NS_2d

\end{lstlisting}

\end{itemize}

\end{document}
