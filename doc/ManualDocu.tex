\listfiles

% allgem. Dokumentenformat
\documentclass[a4paper,11pt,headsepline]{scrartcl}

% weitere Pakete
% Grafiken aus PNG Dateien einbinden
\usepackage{graphicx}

% deutsche Silbentrennung
\usepackage[english]{babel}

% Eurozeichen einbinden
\usepackage[right]{eurosym}

% Umlaute unter UTF8 nutzen
\usepackage[utf8]{inputenc}

% Zeichenencoding
\usepackage[T1]{fontenc}

\usepackage{lmodern}
\usepackage{fix-cm}


% Unterstützung für Schriftarten
%\newcommand{\changefont}[3]{ 
%\fontfamily{#1} \fontseries{#2} \fontshape{#3} \selectfont}

% Paket für Seitenrandabständex und Einstellung für Seitenränder
\usepackage{geometry}
\geometry{left=3.5cm, right=2cm, top=2.5cm, bottom=3cm}

% Paket für Boxen im Text
\usepackage{fancybox}

% bricht lange URLs "schoen" um
\usepackage[hyphens,obeyspaces,spaces]{url}

% Paket für Textfarben
\usepackage{color}

% Mathematische Symbole importieren
\usepackage{amssymb}
\usepackage{amsthm}
\usepackage[intlimits]{amsmath}
\newtheorem{satz}{Satz}%[section]
% auf jeder Seite eine Überschrift (alt, zentriert)
%\pagestyle{headings}

% erzeugt Inhaltsverzeichnis mit Querverweisen zu den Kapiteln (PDF Version)
%\usepackage[bookmarksnumbered,pdftitle={\titleDocument},hyperfootnotes=false]{hyperref} 
%\hypersetup{colorlinks, citecolor=red, linkcolor=blue, urlcolor=black}
%\hypersetup{colorlinks, citecolor=black, linkcolor= black, urlcolor=black}

% neue Kopfzeilen mit fancypaket
\usepackage{fancyhdr} %Paket laden
\pagestyle{fancy} %eigener Seitenstil
\fancyhf{} %alle Kopf- und Fußzeilenfelder bereinigen
\fancyhead[L]{\nouppercase{\leftmark}} %Kopfzeile links
\fancyhead[C]{} %zentrierte Kopfzeile
\fancyhead[R]{\thepage} %Kopfzeile rechts
\renewcommand{\headrulewidth}{0.4pt} %obere Trennlinie
%\fancyfoot[C]{\thepage} %Seitennummer
%\renewcommand{\footrulewidth}{0.4pt} %untere Trennlinie

% für Tabellen
\usepackage{array}

% Runde Klammern für Zitate
%\usepackage[numbers,round]{natbib}

% Festlegung Art der Zitierung - Havardmethode: Abkuerzung Autor + Jahr
%\usepackage{bibgerm}
%\usepackage{natbib}
\bibliographystyle{gerplain}
\usepackage{cite}
% Schaltet den zusätzlichen Zwischenraum ab, den LaTeX normalerweise nach einem Satzzeichen einfügt.
%\frenchspacing

% Paket für Zeilenabstand
\usepackage{setspace}

% für Bildbezeichner
%\usepackage{capt-of}

% für Stichwortverzeichnis
\usepackage{makeidx}

% für Listings
\usepackage{listings}
\lstset{numbers=left, numberstyle=\tiny, numbersep=5pt, keywordstyle=\color{black}\bfseries, stringstyle=\ttfamily,showstringspaces=false,basicstyle=\footnotesize,captionpos=b}
\lstset{language=java}

% Indexerstellung
\makeindex

% Abkürzungsverzeichnis
\usepackage[german]{nomencl}
\let\abbrev\nomenclature

% Abkürzungsverzeichnis LiveTex Version
\renewcommand{\nomname}{Abkürzungsverzeichnis}
\setlength{\nomlabelwidth}{.25\hsize}
\renewcommand{\nomlabel}[1]{#1 \dotfill}
\setlength{\nomitemsep}{-\parsep}
\makenomenclature
%\makeglossary

%SOPHIES SACHEN
\usepackage{textcomp}
\usepackage{eurosym}
\usepackage{amsthm}
\usepackage[intlimits]{amsmath}
\usepackage{amssymb}
\usepackage{amsfonts}
%\usepackage{dsfont}
%\usepackage{mathbbol}
%\usepackage{paralist}
%\usepackage[utf8]{inputenc}
%\usepackage{graphicx}
%\usepackage{subfig}
%\usepackage[T1]{fontenc} 
%\usepackage{biblatex}
%\usepackage{bibgerm}
%\usepackage{indentfirst}
%\usepackage{caption}
%\usepackage{boxedminipage}
%\usepackage{empheq}
%\usepackage{algorithmicx}
%\usepackage[ruled]{algorithm}
%    \usepackage{algpseudocode}
%  \interfootnotelinepenalty=10000  
%    \usepackage[T1]{fontenc}    
%\usepackage{framed}
%\usepackage{enumitem} 
%\usepackage[]{color}
%\definecolor{DarkGreen}{rgb}{0.1,0.7,0.3}   %define a custom color
%\newcommand{\cblue}[1]{\textcolor{blue}{#1}}
%\newcommand{\cred}[1]{\textcolor{red}{#1}}
%\newcommand{\cgreen}[1]{\textcolor{DarkGreen}{#1}} 
\newcommand{\mb}{\mathbf}
\newcommand*\widefbox[1]{\fbox{\hspace{7em}#1\hspace{9em}}}

\newcommand{\diff}{\ \mathrm{d}}
%\usepackage{epigraph}

\renewcaptionname{english}{\figurename}{\textsc{Figure}}
\renewcaptionname{english}{\tablename}{\textsc{Table}}
%\textwidth 16.5cm \textheight 22cm 
\oddsidemargin 0mm
\evensidemargin -4.5mm
\topmargin -10mm
\parindent 10pt


%NORMALE SCHRIFT
\setkomafont{disposition}{\normalcolor\bfseries}
%Linie unter Bild
\newcommand{\decoRule}{\rule{.8\textwidth}{.4pt}}

%============================TITELSEITE============================

\newcommand{\HRule}{\rule{\linewidth}{0.5mm}} % New command to make the lines in the title page
\newcommand*{\supervisor}[1]{\def\supname{#1}}
\newcommand*{\thesistitle}[1]{\def\@title{#1}\def\ttitle{#1}}
\newcommand*{\examiner}[1]{\def\examname{#1}}
\newcommand*{\degree}[1]{\def\degreename{#1}}
\renewcommand*{\author}[1]{\def\authorname{#1}}
\newcommand*{\addresses}[1]{\def\addressname{#1}}
\newcommand*{\university}[1]{\def\univname{#1}}
\newcommand*{\department}[1]{\def\deptname{#1}}
\newcommand*{\group}[1]{\def\groupname{#1}}
\newcommand*{\faculty}[1]{\def\facname{#1}}
\newcommand*{\subjects}[1]{\def\subjectname{#1}}
\newcommand*{\keywords}[1]{\def\keywordnames{#1}}
\def\authorname{}
\def\ttitle{}


%===================================================================
\usepackage{array}
\newcolumntype{L}[1]{>{\raggedright\let\newline\\\arraybackslash\hspace{0pt}}m{#1}}

\lstset{language=[90]Fortran,
  basicstyle=\ttfamily,
  keywordstyle=\color{blue},
  breaklines=true,
  %commentstyle=\color{green},
  %morecomment=[l]{!\ }% Comment only with space after !
}

\begin{document}
\title{What to keep in mind while commenting/documenting the code}
\maketitle
\onehalfspacing
\noindent First of all: This is meant as a short guidance. A detailed description of Doxygen can be found on http://www.stack.nl/~dimitri/doxygen/index.html \\
\section{Doxygen's configuration file}
The most important configuration options I could find out so far: \\
\hfill \\
\begin{tabular}{|c|L{5.5cm}|L{4.6cm}|}
\hline 
Configuration option & Explanation & Example \\ 
\hline 
PROJECT\_NAME & This name is used in the title of most generated pages and in a few other places &  \textsc{project\_name = \grqq wabbit\grqq} \\
\hline
PROJECT\_BRIEF &  optional one line description for a project that appears at the top of each page & \textsc{project\_brief} = \grqq (W)avelet (A)daptive (B)lock-(B)ased solver for (I)nsects in (T)urbulence\grqq \\
\hline
OUTPUT\_DIRECTORY &  Specify the path into which the generated documentation will be written. &  \textsc{output\_directory} = ./doc/output 
\\
\hline
OPTIMIZE\_FOR\_FORTRAN & Doxygen will generate output that is tailored for Fortran. & \textsc{optimize\_for\_fortran = yes} \\
\hline
EXTENSION\_MAPPING & Doxygen selects the parser to use depending on the extension of the files it parses. & \textsc{extension\_mapping }= f90=FortranFree \\
\hline
MARKDOWN\_SUPPORT & Doxygen pre-processes all comments according to the Markdown format, which allows for more readable documentation (see section \ref{markdown}) & \textsc{markdown\_support = yes} \\
\hline
\end{tabular}
\begin{tabular}{|c|L{5.5cm}|L{4.6cm}|}
\hline
Configuration option & Explanation & Example \\ 
\hline
EXTRACT\_ALL & If set to yes, doxygen will assume all entities in the documentation are documented, even if no documentation was available. & \textsc{extract\_all = no}\\
\hline
GENERATE\_TODOLIST & This list is created by putting \textbackslash todo commands in the documentation.& \textsc{generate\_todolist = yes}\\
\hline
INPUT & Specify the files and/or directories that contain documented source files. & \textsc{input} = ./LIB \textbackslash \hfill
		       	 ./doc/mainpage.txt \\
\hline
FILE\_PATTERNS & If value of INPUT tag contains directories, the FILE\_PATTERNS tag specifies one or more wildcard patterns to filter out the source-files in the directories. & \textsc{file\_patterns} = *.f90 \\
\hline
RECURSIVE & To specify whether or not subdirectories should be searched for input files as well. & \textsc{recursive = yes} \\
\hline
IMAGE\_PATH & To specify one or more files or directories that contain images that are to be included in the documentation (see section \ref{image})&  \textsc{image\_path} = ./doc/images \\
 \hline
SOURCE\_BROWSER & A list of source files will be generated. Documented entities will be cross-referenced with these sources.& \textsc{source\_browser = yes} \\
\hline
REFERENCES\_LINK\_SOURCE & Hyperlinks from functions in REFERENCES\_RELATION and REFERENCED\_BY\_RELATION lists will link to the source code. & \textsc{references\_link\_source = yes} \\
\hline
HTML\_OUTPUT & Used to specify where the HTML docs will be put. Relative path: Value of \textsc{output\_directory} is put in front of it. (Same for latex output) & \textsc{html\_output} = html \\
\hline
\end{tabular} 
\begin{tabular}{|c|L{5.5cm}|L{4.6cm}|}
\hline 
Configuration option & Explanation & Example \\ 
\hline
HTML\_EXTRA\_STYLESHEET & Using this option one can overrule certain style aspects. & \textsc{html\_extra\_stylesheet} =  ./doc/mystyle.css \\
\hline
MAX\_DOT\_GRAPH\_DEPTH &  Set the maximum depth of the graphs generated by dot (e.g. the call graphs) & \textsc{max\_dot\_graph\_depth} = 1\\
\hline 
\end{tabular} 
\section{Comments}
For Fortran code, \grqq !>\grqq \ or \grqq !<\grqq \ starts a comment and \grqq !!\grqq \ or \grqq !>\grqq \ can be used to continue an one line comment into a multi-line comment. Note that in-body documentation is not implemented for Fortran! An example for a comment block:
\lstinputlisting[language=Fortran, linerange={12-14}]{../LIB/MESH/compute_friends_table.f90}
\subsection*{Parameters}
\lstinputlisting[language=Fortran, linerange={32-41}]{../LIB/MESH/update_neighbors_2D.f90}
\section{Special Commands} \label{image}
Here are some special commands you should include while commenting new code:
\begin{description}
\item[\texttt{!> \textbackslash file}]\hfill \\
Indicates that a comment block contains documentation for a source or header file. If \textsc{extract\_all} is set to \textsc{no}, you will need this to include the file into the documentation.
\item[\texttt{!> \textbackslash callgraph}]\hfill \\ 
If put in a comment block of a function or method and \textsc{have\_dot} set to \textsc{yes}, then doxygen will generate a call graph for that function (provided the implementation of the function or method calls other documented functions).
\item[\texttt{!> \textbackslash mainpage}]\hfill \\ 
If the \textbackslash mainpage command is placed in a comment block the block is used to customize the index page (in HTML) or the first chapter (in Latex).
\item[\texttt{!> \textbackslash name}]\hfill \\ 
Name of the file/subroutine.
\item[\texttt{!> \textbackslash version}]\hfill \\
Current version of subroutine/file.
\item[\texttt{!> \textbackslash author}]\hfill \\ 
Author(s) of the subroutine/file.
\end{description}
\lstinputlisting[language=Fortran, linerange={1-2, 4-4, 6-9}]{../LIB/MESH/treecode_to_hilbertcode_2D.f90}
\begin{description}
\item[\texttt{!> \textbackslash brief}]\hfill \\ 
Starts a paragraph that serves as a brief description.
\item[\texttt{!> \textbackslash details}]\hfill \\ 
Just like \texttt{!> \textbackslash brief} starts a brief description, \texttt{!> \textbackslash details} starts the detailed description. You can also start a new paragraph with a blank line.
\item[\texttt{!> \textbackslash image}] \hfill \\ 
Inserts an image into the documentation. This command is format specific. First argument:  output format in which the image should be embedded, second argument: file name of the image, third argument: caption, fourth argument: width or height of the image.
\end{description}
\lstinputlisting[language=Fortran, firstline=39, lastline=39]{../LIB/MESH/treecode_to_hilbertcode_2D.f90}
\begin{description}
\item[\texttt{!> \textbackslash note}]\hfill \\ 
Starts a paragraph where a note can be entered. The paragraph will be indented.
\item[\texttt{!> \textbackslash todo}]\hfill \\
 Starts a paragraph where a \textsc{todo} item is described.
\end{description}
Both the \texttt{!> \textbackslash note} and the \texttt{!> \textbackslash todo} paragraph will be highlighted.
\section{Markdown support} \label{markdown}
You can use Markdown formatting syntax to let the output look good.
\subsection{Paragraphs}
Paragraphs are seperated by a blank line. To start a new line, put \grqq \textbackslash n\grqq at the end of the line.
\subsection{Tables} 
A table consists of a header line, a separator line, and at least one row line. Table columns are separated by the pipe (|) character.
\lstinputlisting[language=Fortran, firstline=22, lastline=26]{../LIB/MESH/treecode_to_hilbertcode_2D.f90}
\subsection{Lists}
Simple bullet lists can be made by starting a line with -, +, or *. List items can span multiple paragraphs (if each paragraph starts with the proper indentation) and lists can be nested. To start with a new item include a blank line.
\lstinputlisting[language=Fortran, firstline=12, lastline=19]{../LIB/MESH/treecode_to_hilbertcode_2D.f90}
You can also generate a numbered list:
\lstset{language=Fortran}
\begin{lstlisting}
!! 1. First item
!! 2. Second item
\end{lstlisting}
\subsection{Automatic Linking}
To create a link to an URL or e-mail address Markdown supports the following syntax:
\lstinputlisting[language=Fortran, firstline=6, lastline=6]{mainpage.txt}
Note that ` `   lets the text appear in monospaced font.
\end{document}